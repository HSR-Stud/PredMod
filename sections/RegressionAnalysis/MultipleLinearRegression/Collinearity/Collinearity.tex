\subsubsection{Collinearity}
	\begin{itemize}
	    \item Collinearity refers to the situation in which two or more predictor variables are closely related to one another. 
	    \item Collinearity causes the standard error of the fitted coefficients to grow.
	    \item Since for the t-statistic, the coefficients are divided by their standard errors, we may fail to reject the null hypothesis.
	\end{itemize}
	
	\paragraph{Variance Inflation Factor (VIF)}
		
		\RTheory
		{
			The ratio of the variance of a coefficient when fitting the full model, divided by the variance of the coefficient when fitted on its own.
			
			$$ \mathrm{VIF}(\beta_j) = \frac{1}{1-R^2_{X_j|X_{-j}}}$$
			
			The minimum value of the VIF is 1, which indicates no collinearity.
			
			\textbf{Rule of thumb:}
			
			A $\mathrm{VIF}>5$ indicates a problematic amount of collinearity
		}
		{
			sections/RegressionAnalysis/MultipleLinearRegression/Collinearity/VIFExample.R
		}