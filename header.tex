
% Genereller Header
\documentclass[10pt,twoside,a4paper,fleqn]{article}
\usepackage[utf8x]{inputenc}
\usepackage[left=1cm,right=1cm,top=1cm,bottom=1cm,includeheadfoot]{geometry}
\usepackage[ngerman]{babel,varioref}
\usepackage[T1]{fontenc}


% Pakete
\usepackage{amssymb} 
\usepackage{amsmath}
\usepackage{fancybox}
\usepackage{graphicx}
\usepackage{color}
\usepackage{lastpage}
\usepackage{wrapfig}
\usepackage{fancyhdr}
\usepackage{hyperref}
\usepackage{verbatim}
\usepackage{floatflt}
\usepackage{arydshln}
\usepackage{ucs}
\usepackage{pdflscape} % landscape
\usepackage{multirow} % zellen in tabellen verbinden
\usepackage{multicol} 
%\usepackage{pgf,tikz} %Zeichnen

% \usepackage{slashbox} % getrennte zelle in tabelle
% \usepackage{array} % anordnung in tabellen

%%%%%%%%%%%%%%%%%%%%
% Generelle Makros %
%%%%%%%%%%%%%%%%%%%%


\newcommand{\skript}[1]{$_{\textcolor{red}{\mbox{\small{Script p. #1}}}}$}
\newcommand{\sachs}[1]{$_{\textcolor{blue}{\mbox{\small{Sachs S. #1}}}}$}
\newcommand{\formelbuch}[1]{$_{\textcolor{red}{\mbox{\small{S#1}}}}$}
\newcommand{\verweis}[2]{ {\small (siehe auch \ref{#1}, #2 (S. \pageref{#1}))}}
\newcommand{\subsubadd}[1]{\textcolor{black}{\mbox{#1}}}
\newenvironment{liste}[0]{\begin{list}{$\bullet$}{\setlength{\itemsep}{0cm}\setlength{\parsep}{0cm} \setlength{\topsep}{0cm}}}{\end{list}}
    
\newcommand{\logd}[0]{\log_{10}}
\newcommand{\subsubsubsection}[1]{\textbf{#1}}

\newenvironment{aufzaehlung}[0]{\begin{enumerate}{\setlength{\itemsep}{0cm}\setlength{\parsep}{0cm}\setlength{\topsep}{0cm}}} {\end{enumerate}}

\newcommand{\abbHeight}[3]{
	\begin{center}
		\includegraphics[height=#2]{./bilder/#1} \\
		#3
    \end{center}
}


\newcommand{\skriptsection}[2]{\section{#1 \formelbuch{#2}}}
\newcommand{\skriptsubsection}[2]{\subsection{#1 \formelbuch{#2}}}
\newcommand{\skriptsubsubsection}[2]{\subsubsection{#1 \formelbuch{#2}}}

%%%%%%%%%%
% Farben %
%%%%%%%%%%
\definecolor{black}{rgb}{0,0,0}
\definecolor{red}{rgb}{1,0,0}
\definecolor{white}{rgb}{1,1,1}
\definecolor{grey}{rgb}{0.8,0.8,0.8}

%%%%%%%%%%%%%%%%%%%%%%%%%%%%
% Mathematische Operatoren %
%%%%%%%%%%%%%%%%%%%%%%%%%%%%
\DeclareMathOperator{\sinc}{sinc}
\DeclareMathOperator{\sgn}{sgn}



% Fouriertransformationen
\unitlength1cm
\newcommand{\FT}
{
\begin{picture}(1,0.5)
\put(0.2,0.1){\circle{0.14}}\put(0.27,0.1){\line(1,0){0.5}}\put(0.77,0.1){\circle*{0.14}}
\end{picture}
}


\newcommand{\IFT}
{
\begin{picture}(1,0.5)
\put(0.2,0.1){\circle*{0.14}}\put(0.27,0.1){\line(1,0){0.45}}\put(0.77,0.1){\circle{0.14}}
\end{picture}
}
\newcommand{\jw}{j\omega}

\newcommand{\DFT}
{
%\overset{DFT}{
	\begin{picture}(1,0.2)
	\put(0.2,0.1){\circle{0.14}}{\put(0.27,0.1){\line(1,0){0.5}}}\put(0.77,0.1){\circle*{0.14}}
	\end{picture}
%}
}

\newcommand{\IDFT}
{
%\overset{IDFT}{
    \begin{picture}(1,0.2)
	\put(0.2,0.1){\circle*{0.14}}\put(0.27,0.1){\line(1,0){0.45}}\put(0.77,0.1){\circle{0.14}}
	\end{picture}
%}
}


\newcommand{\twopartdef}[4]
{
	\left\{
		\begin{array}{ll}
			#1 & \mbox{if } #2 \\
			#3 & \mbox{if } #4
		\end{array}
	\right.
}

\newcommand{\mtwopartdef}[3]
{
	\left\{
		\begin{array}{ll}
			#1 & \mbox{if } #2 \\
			#3 & \mbox{otherwise }
		\end{array}
	\right.
}

\newcommand{\threepartdef}[6]
{
	\left\{
		\begin{array}{lll}
			#1 & \mbox{if } #2 \\
			#3 & \mbox{if } #4 \\
			#5 & \mbox{if } #6
		\end{array}
	\right.
}

\newcommand{\bm}{\boldsymbol}
\newcommand{\todo}[2][red]{\textcolor{#1}{TODO: #2}}
\newcommand{\numbercircled}[1]{\textcircled{\raisebox{-1pt}{#1}}}



%%%%%%%%%%%%%%%%%%%%%%%%%%%%
% Allgemeine Einstellungen %
%%%%%%%%%%%%%%%%%%%%%%%%%%%%
%pdf info
\hypersetup{pdfauthor={\authorinfo},pdftitle={\titleinfo},colorlinks=false}
\author{\authorinfo}
\title{\titleinfo}

%Kopf- und Fusszeile
\pagestyle{fancy}
\fancyhf{}
%Linien oben und unten
\renewcommand{\headrulewidth}{0.5pt} 
\renewcommand{\footrulewidth}{0.5pt}


\fancyhead[L]{\titleinfo{ }- Summary}
%Kopfzeile rechts bzw. aussen
\fancyhead[R]{\today{ }- Page \thepage/\pageref{LastPage}}
\fancyfoot[C]{\copyright{ }\authorinfo}

% Einrücken verhindern versuchen
\setlength{\parindent}{0pt}


% Einheiten
%#########################################################################################
\usepackage[Gray,squaren]{SIunits} %\gray befehl heisst nun \Gray

%Spannung
\DeclareMathOperator{\V}{\volt}
\DeclareMathOperator{\mV}{\milli \volt}
\DeclareMathOperator{\uV}{\micro \volt}

%Strom
\DeclareMathOperator{\A}{\ampere}
\DeclareMathOperator{\mA}{\milli \ampere}
\DeclareMathOperator{\uA}{\micro \ampere}
\DeclareMathOperator{\nA}{\micro \ampere}

%Zeit
\DeclareMathOperator{\s}{\second}
\DeclareMathOperator{\ms}{\milli \second}
\DeclareMathOperator{\us}{\micro \second}
\DeclareMathOperator{\ns}{\nano \second}

%Kapazität
\DeclareMathOperator{\mF}{\milli \farad}
\DeclareMathOperator{\uF}{\micro \farad}
\DeclareMathOperator{\nF}{\nano \farad}
\DeclareMathOperator{\pF}{\pico \farad}
\DeclareMathOperator{\fF}{\femto \farad}

%Induktivität
\DeclareMathOperator{\mH}{\milli \henry}
\DeclareMathOperator{\uH}{\milli \henry}
\DeclareMathOperator{\nH}{\nano \henry}

%Widerstand
\DeclareMathOperator{\MO}{\mega \ohm}
\DeclareMathOperator{\kO}{\kilo \ohm}
\DeclareMathOperator{\mO}{\milli \ohm}

%Strecke
\DeclareMathOperator{\km}{\kilo \meter}
\DeclareMathOperator{\cm}{\centi \meter}
\DeclareMathOperator{\mm}{\milli \meter}

%Frequenz
\DeclareMathOperator{\GHz}{\giga \hertz}
\DeclareMathOperator{\MHz}{\mega \hertz}
\DeclareMathOperator{\Hz}{\hertz}
\DeclareMathOperator{\kHz}{\kilo \hertz}
\DeclareMathOperator{\mHz}{\milli \hertz}

%Leistung
\DeclareMathOperator{\kW}{\kilo \watt}
\DeclareMathOperator{\mW}{\milli \watt}
\DeclareMathOperator{\uW}{\micro \watt}

%Kreisfrequenz
\DeclareMathOperator{\rpers}{\radianpersecond}

%DeziBel
\DeclareMathOperator{\dB}{\deci \bel}

\DeclareMathOperator{\e}{\mathrm e}


